\section*{Attribution Mechanism (Addendum)}
We propose an attribution mechanism to increase auditability of CPPTAI:

\textbf{What to Track}
- Which blocks caused the solution state $S$ to increase or decrease at each floor.
- Counterfactual analysis: estimates of $S$ if a given floor were skipped.
- Visual explanation: descent trajectory with annotated decision points.

\textbf{Implementation}
- Extend \texttt{ProblemBlock} with an \texttt{influence\_score}.
- Track gradient attributions during Phase~III (Cognitive Descent).
- Generate a human-readable explanation alongside the final solution output.

\textbf{Impact}
Strengthens auditability for high-stakes domains (healthcare, legal, finance) where black-box reasoning is unacceptable.

\section*{Latest Results}
With \texttt{BENCH\_DISABLE\_EXTERNAL=1}, the benchmark summary is:
- CoT: accuracy $0.0$, error $1.0$, diversity $\approx 0.992$, time $0.0$ s.
- ToT: accuracy $0.1$, error $0.9$, diversity $1.0$, time $0.0$ s.
- GoT: accuracy $0.0$, error $1.0$, diversity $1.0$, time $0.0$ s.
- ReAct: accuracy $0.1$, error $0.9$, diversity $\approx 0.985$, time $0.0$ s.
- CPPTAI: accuracy $1.0$, error $0.0$, diversity $\approx 0.992$, robust diversity $\approx 0.812$, clusters $3$.
- CPPTAI\_no\_IV: accuracy $1.0$, error $0.0$, same diversity/robustness; time slightly different.
- CPPTAI\_no\_I: accuracy $1.0$, error $0.0$, same diversity/robustness; time slightly different.

Paired comparisons in \texttt{stats\_summary.csv} include $t$-statistic, Cohen's $d$, and an approximate two-sided $p$-value; ablation comparisons currently yield near-zero $t$/$d$ and $p\approx 1$ due to identical accuracy.
